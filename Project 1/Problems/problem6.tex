% Håndterer dokumentklasse, språk og symboler
\documentclass[final, 3p, times, 11.5pt]{article}

\usepackage[english]{babel}
\usepackage[utf8]{inputenc}
\usepackage[T1]{fontenc}

% Dokumentformatering:
\setlength\parindent{0pt} %ingen innrykk
\usepackage[parfill]{parskip} % Mellomrom mellom avsnitt
% Endre str
\usepackage[a4paper,top=3cm,bottom=2cm,left=1.5cm,right=1.5cm,marginparwidth=1.75cm]{geometry} % utnytter større del av arket.
\usepackage{appendix}
\usepackage{multicol}
\usepackage{titlesec}
\titleformat{\chapter}[display]
  {\normalfont\huge\bfseries}{\chaptertitlename\ \thechapter}{20pt}{\Huge}
\titlespacing*{\chapter}{0pt}{-70pt}{20pt} % reduce headspace of the chapter title
\setcounter{secnumdepth}{3} % Set the depth of section numbering to subsubsections



% Matte og symboler
\usepackage{lmodern}
\usepackage{textcomp, gensymb}
\usepackage{amssymb}
\usepackage{amsmath}
\usepackage{mathtools}
\usepackage{physics}
\usepackage{cancel} % Gjennomstreking i likninger etc
\usepackage{amsthm}
\usepackage{caption}
\usepackage{dirtytalk}
\usepackage{cancel}

% Pseudocode
\usepackage{algorithm}
\usepackage{algpseudocode}

% Håndterer grafikk:
\usepackage{graphicx}
\graphicspath{{Figurer/}} % Henter grafikk fra mappen "Figurer"
\usepackage{wrapfig}
%\setlength{\intextsep}{2pt} % adjust vertical space above and below the wrapfigure
\usepackage{subcaption} % eller subfigure
\usepackage[font=small,labelfont=bf]{caption} % Mindre figurtekst, bold på figur numerering
\usepackage{transparent} % Kan gjøre tekst mere gjenomsiktig
\usepackage{listings}
\usepackage{paralist}
\usepackage{tikz}
\usetikzlibrary{matrix}
\usetikzlibrary{fit}
\usetikzlibrary{intersections}
\usepackage{multirow}
\usepackage{pdfpages} % sett in pdf i dokumentet
\usepackage{subcaption} 
\usepackage{tabularx}
\usepackage{tabularray} % beste tabellpakke
\usepackage{pgfplots}
\usepgfplotslibrary{fillbetween}
\pgfplotsset{width=10cm,compat=1.9}
\newcommand{\tabitem}{~~\llap{\textbullet}~~} % for kulepunkter i tabeller uten vspace over dem


% Håndterer fine tabeller:
\usepackage{booktabs}
\usepackage{multirow}

% For å hente inn pdfsider. 
\usepackage{pdfpages}

% Håntere kilder med hyperlinker
\usepackage{csquotes}
\usepackage{hyperref}
\hypersetup{
    colorlinks=true,        % Colored links instead of frames
    linkcolor=blue,         % Color for internal links
    citecolor=red,          % Color for citations
    urlcolor=red            % Color for external links
}
\usepackage[style=numeric, sorting=none]{biblatex} 
\addbibresource{sources.bib}
\usepackage{lastpage} 
\usepackage[printonlyused,withpage]{acronym}
\usepackage{footmisc}




% Egne funksjoner
\newcommand{\frontmatter}{\cleardoublepage \pagenumbering{roman}}
\newcommand{\mainmatter}{\cleardoublepage \pagenumbering{arabic}}

\usepackage{caption}
\captionsetup{%
    ,format=hang
    ,justification=raggedright
    ,singlelinecheck=false
    ,figureposition=top
    }


\makeatletter
\AtBeginDocument{%
  \renewcommand*{\AC@hyperlink}[2]{%
    \begingroup
      \hypersetup{hidelinks}%
      \hyperlink{#1}{#2}%
    \endgroup
  }%
}
\makeatother





% Håndterer multi-fil oppsett:
\usepackage{xr}
\usepackage{subfiles}
\externaldocument{\subfix{main}}
 % Henter inn pakker og forhåndsdefinerte instillinger

\begin{document}
\section*{Problem 6}

\textbf{a)}

Since $A$ is a general tridiagonal matrix we can use the \textit{Thomas algorithm} which consists of; 1) a forward substitution, and 2) a backward substitution on the augmented matrix $B = \begin{bmatrix}A & \mathbf{g} \end{bmatrix} $. The end result is an augmented matrix $\begin{bmatrix}I & \mathbf{v} \end{bmatrix}$. A general tridiagonal matrix $A \in \mathbb{R}^{n\times n}$ has the form 

$$
\begin{bmatrix}
b_1 & c_1 & 0 & \dots & 0 \\
a_2 & b_2 & c_2 & \dots & 0 \\
0   & a_3 & b_3 & \dots & 0 \\
\vdots & \vdots & \vdots & \ddots & \vdots \\
0  & 0 & 0 & a_n & b_n 
\end{bmatrix} \quad ,
$$

where the subdiagonal is formed by $\mathbf{a} = [a_2, a_3, ..., a_n]$, the diagonal is formed by $\mathbf{b} = [b_1, b_2, ..., b_n]$ and the superdiagonal is formed by $\mathbf{c} = [c_1, c_2, ..., c_{n-1}]$. Note here that first element of $\mathbf{a}$ is found in row number two ($R_2$) and the last element of $\mathbf{c}$ is found in the second to last row ($R_{n-1}$).

During the forward substitution (fsub) we iterate through the rows $R_2, R_3, ..., R_n$ of the augmented matrix $[A \mathbf{g}]$ and performing the row operation $R_j \to R_j - \frac{a_j}{b_{j-1}}R_{j-1}$ for $j=2, 3, ... n$. The result is that $B$ becomes upper-traigonal on the form 

$$
\begin{bmatrix}A & \mathbf{g} \end{bmatrix} \stackrel{\text{fsub}}{\to}
\left[
\begin{array}{ccccc|c}
\tilde{b}_1 & c_1 & 0 & \dots & 0 & \tilde{g}_1\\
0 & \tilde{b}_2 & c_2 & \dots & 0 & \tilde{g}_2\\
0 & 0 & \tilde{b}_3 & \dots & 0 & \tilde{g}_3 \\
\vdots & \vdots & \vdots & \ddots & \vdots & \vdots\\
0  & 0 & 0 & 0 & \tilde{b}_n & \tilde{g}_n
\end{array}
\right] (:= B_{fsub}) \quad ,
$$

where $$
\tilde{b}_j = 
\begin{cases} 
b_j \quad  & \text{, for } j=1 \\
b_j - \frac{a_j}{\tilde{b}_{j-1}}c_{j-1} & \text{, for } j\neq 1
\end{cases}
\quad \text{ and , } \quad
\tilde{g}_j = 
\begin{cases} 
g_j \quad  & \text{, for } j=1 \\
g_j - \frac{a_j}{\tilde{b}_{j-1}}\tilde{g}_{j-1} & \text{, for } j\neq 1
\end{cases} \quad .
$$

During the backwar substitution (bsub) we iterate through the rows $R_n, R_{n-1}, ..., R_1$ of $B_{fsub}$ and perform the row operations $R_n \to \frac{R_n}{\tilde{b}_n}$ and $R_j \to \frac{R_j -c_jR_{j+1}}{\tilde{b}_j}$ for $j=n-1, n-2, ... n_1$. The result is that $B_{fsub}$ becomes $\begin{bmatrix}I & \mathbf{v} \end{bmatrix}$ where
$$
v_j = 
\begin{cases} 
\frac{\tilde{g}_j}{\tilde{b}_n} \quad  & \text{, for } j=n \\
\frac{\tilde{g}_j - c_jv_{j+1}}{\tilde{b}_j} & \text{, for } j\neq n
\end{cases} \quad .
$$

Using this algorithm we will perform (ignoring the operations for $j=1$ in the fsub and $j=n$ in bsub) six FLOPs in the fsub and three FLOPS in the bsub per $j$. We will run each loop for $n-1$ iterations. Adding the inetialization (1 FLOP) this yields 
$$
(9(n-1)+1) = 9n-7 \operatorname{FLOPs} \quad.
$$

For $n\to \infty$ we can approximate the number of FLOPS as $9n$.

\end{document}