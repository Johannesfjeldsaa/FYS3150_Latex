\documentclass[final, 3p, times, 11.5pt]{article}

\usepackage[english]{babel}
\usepackage[utf8]{inputenc}
\usepackage[T1]{fontenc}

% Dokumentformatering:
\setlength\parindent{0pt} %ingen innrykk
\usepackage[parfill]{parskip} % Mellomrom mellom avsnitt
% Endre str
\usepackage[a4paper,top=3cm,bottom=2cm,left=1.5cm,right=1.5cm,marginparwidth=1.75cm]{geometry} % utnytter større del av arket.
\usepackage{appendix}
\usepackage{multicol}
\usepackage{titlesec}
\titleformat{\chapter}[display]
  {\normalfont\huge\bfseries}{\chaptertitlename\ \thechapter}{20pt}{\Huge}
\titlespacing*{\chapter}{0pt}{-70pt}{20pt} % reduce headspace of the chapter title
\setcounter{secnumdepth}{0} % Set the depth of section numbering to subsubsections



% Matte og symboler
\usepackage{lmodern}
\usepackage{textcomp, gensymb}
\usepackage{amssymb}
\usepackage{amsmath}
\usepackage{mathtools}
\usepackage{physics}
\usepackage{cancel} % Gjennomstreking i likninger etc
\usepackage{amsthm}
\usepackage{caption}
\usepackage{dirtytalk}
\usepackage{cancel}

% Pseudocode
\usepackage{algorithm}
\usepackage{algpseudocode}

% Håndterer grafikk:
\usepackage{graphicx}
\graphicspath{{Figures/}} % Henter grafikk fra mappen "Figurer"
\usepackage{wrapfig}
\setlength{\intextsep}{2pt} % adjust vertical space above and below the wrapfigure
\usepackage{subcaption} % eller subfigure
\usepackage[font=small,labelfont=bf]{caption} % Mindre figurtekst, bold på figur numerering
\usepackage{transparent} % Kan gjøre tekst mere gjenomsiktig
\usepackage{listings}
\usepackage{paralist}
\usepackage{tikz}
\usetikzlibrary{matrix}
\usetikzlibrary{fit}
\usetikzlibrary{intersections}
\usepackage{multirow}
\usepackage{pdfpages} % sett in pdf i dokumentet
\usepackage{subcaption} 
\usepackage{tabularx}
\usepackage{tabularray} % beste tabellpakke
\usepackage{pgfplots}
\usepgfplotslibrary{fillbetween}
\pgfplotsset{width=10cm,compat=1.9}
\newcommand{\tabitem}{~~\llap{\textbullet}~~} % for kulepunkter i tabeller uten vspace over dem

% Håndterer fine tabeller:
\usepackage{booktabs}
\usepackage{multirow}

% Håntere kilder med hyperlinker
\usepackage{csquotes}
\usepackage{hyperref}
\hypersetup{
    colorlinks=true,        % Colored links instead of frames
    linkcolor=blue,         % Color for internal links
    citecolor=red,          % Color for citations
    urlcolor=red            % Color for external links
}
\usepackage[style=numeric, sorting=none]{biblatex} 
\addbibresource{sources.bib}
\usepackage{lastpage} 
\usepackage[printonlyused,withpage]{acronym}
\usepackage{footmisc}




% Egne funksjoner
\newcommand{\frontmatter}{\cleardoublepage \pagenumbering{roman}}
\newcommand{\mainmatter}{\cleardoublepage \pagenumbering{arabic}}

\usepackage{caption}
\captionsetup{%
    ,format=hang
    ,justification=raggedright
    ,singlelinecheck=false
    ,figureposition=top
    }


\makeatletter
\AtBeginDocument{%
  \renewcommand*{\AC@hyperlink}[2]{%
    \begingroup
      \hypersetup{hidelinks}%
      \hyperlink{#1}{#2}%
    \endgroup
  }%
}
\makeatother





% Håndterer multi-fil oppsett:
\usepackage{xr}
\usepackage{subfiles}
\externaldocument{\subfix{main}}
 

\begin{document}
\section{Problem 2}

Code solution found in github repo in file \textit{`Project2/solutions/p2.cpp`} and code output will be written to \\ \textit{`Project2/output/problem2data.out`}. 


To set up the tridiagonal matrix $A \in \mathbb{R}^{N\times N}$ we firstly provide the corresponding number of parts $n$ for which we want to split the $x$ domain into. Because of the discretization presented in the project description we therefor provide $n=7 \implies N = n-1 = 6$. Further the values $a =- \frac{1}{h^{2}}$ and $d = \frac{2}{h^{2}}$ are calculated using $h = \frac{1}{n} $. The symmetric tridiagonal matrix will have the vector $\mathbf{d} = \begin{bmatrix} 2/h^{2} & 2/h^{2} & \dots & 2/h^{2} \end{bmatrix}$ on the sub- and super- diagonal and the vector $\mathbf{a} = \begin{bmatrix} -1/h^{2} & -1/h^{2} & \dots & -1/h^{2} \end{bmatrix}$ on the diagonal. The resulting matrix becomes 

$$
A = \frac{1}{h^{2}}
\begin{bmatrix}
2   &   -1  &   0   &   \dots   &   0   &   0   \\
-1  &   2   &   -1  &   \dots   &   0   &   0   \\
0   &   -1  &   2   &   \dots   &   0   &   0   \\
\vdots & \vdots & \vdots    &\ddots & \vdots & \vdots\\
0   &   0   &   0   &   \dots   &   2 & -1   \\   
0   &   0   &   0   &   \dots   &  -1 & 2 
\end{bmatrix} \stackrel{N=6}{\implies}
\begin{bmatrix}
   98.0 & -49.0 &        0 &        0  &       0  &       0 \\
  -49.0 &  98.0 & -49.0 &        0   &      0  &       0 \\
         0 & -49.0  & 98.0  &-49.0 &        0    &     0 \\
         0 &        0  &-49.0  & 98.0  &-49.0   &      0 \\
         0 &        0  &       0  &-49.0   &98.0  &-49.0 \\
         0 &        0  &       0  &       0  &-49.0 &  98.0
\end{bmatrix}
$$

The analytical eigenpares are calculated using the equations provided in project description and match with the eigenpairs calculated by armadillo's \textit{`eig\_sym`} function which uses the eigendecomposition of $A$ to find the eigenpairs: 

\textbf{Eigenvalues:}
    \begin{align*}
    \mathbf{\lambda} &= \begin{bmatrix} \lambda^{(1)} & \lambda^{(2)} & \dots & \lambda^{(6)}\end{bmatrix}        \\
    &= \begin{bmatrix} 
    9.7051 & 36.8980 & 76.1930 & 119.8100 & 159.1000 & 186.2900
    \end{bmatrix}
    \end{align*}

\textbf{Eigenvectors:}

    \begin{align*}
        v &= \begin{bmatrix}\mathbf{v^{(1)}} & \mathbf{v^{(2)}} & \dots & \mathbf{v^{(6)}}\end{bmatrix} \\
        & = \begin{bmatrix}
            0.2319 & -0.4179 & 0.5211 & 0.5211 & 0.4179 & -0.2319 \\
0.4179 & -0.5211 & 0.2319 & -0.2319 & -0.5211 & 0.4179 \\
0.5211 & -0.2319 & -0.4179 & -0.4179 & 0.2319 & -0.5211 \\
0.5211 & 0.2319 & -0.4179 & 0.4179 & 0.2319 & 0.5211 \\
0.4179 & 0.5211 & 0.2319 & 0.2319 & -0.5211 & -0.4179 \\
0.2319 & 0.4179 & 0.5211 & -0.5211 & 0.4179 & 0.2319 \\
        \end{bmatrix}
    \end{align*}

\end{document}
