\documentclass[../main_proj3.tex]{subfiles}

\graphicspath{{\subfix{Figures/}}}

\begin{document}

\section{Conclusion}\label{sec:conclusion}

In this report we have investigated the use of numerical methods to simulate a Penning trap. Firstly we benchmarked our simulation against analytical results showing a good ability to reproduce positions after a simulation time of 50 microseconds. The RK4 method was the more stable method, but both algorithm showed increased numerical stability as the number of steps increased. Further the role of Coloumb forces was investigated showing that the simulator indeed is able to produce interacting particles affecting the periodic trajectories of both particle 1 and 2. Lastly we showed how the Penning trap stability can be compromised when the ratio of magnetic field strength to applied potential is altered. To conclude we have successfully implemented a numeric penning trap. For future work a timing of different numerical simulation techniques should be investigate to quantify which method yields the lowest error per time used for simulation.

\end{document}