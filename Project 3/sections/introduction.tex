\documentclass[../main_proj3.tex]{subfiles}

\graphicspath{{\subfix{Figures/}}}

\begin{document}
\section{Introduction}

The standard model of physics is perhaps one of the greater achievements of physics and has allowed us to gain a wider understanding of the fundamental principles of our universe. It does so by describing how the building blocks, quarks and leptons, make up all known matter and how the force-carrying bosons influence the quarks and leptons \cite{jaffe2018physics}. The last piece of the model, the Higgs boson, was discovered in 2012, confirming \textit{how} the particles acquire mass \cite{aad2012observation}. However, the standard model is not complete, and to further test and refine our understanding of the standard model, high-precision experimental measurements are needed.

One of the technologies that enables us to make these measurements is the Penning trap \cite{ulmer2018precision}. To investigate this technology, this report will look into simulating a Penning trap. The purpose being to understand how charged particles interact with the electro-magnetic field of the trap and look into the effect of adding multiple charged particles with respect to system stability.

In Section \textit{methods and theory} I describe the mathematical background for the Penning Trap and explain how to solve the governing equation for the position and velocity both analytically for the special case of a single particle system and numerically for multiple interacting particles. In Section \textit{results and discussion}, central results for validating code functionality and investigating the resonance frequencies of the system are presented and discussed. Lastly, Section \textit{conclusion} provides a short summary of the project.

All code developed for this report is available via a GitHub repository \href{github.com/Johannesfjeldsaa/FYS3150}{https://github.com/Johannesfjeldsaa/FYS3150}, and \LaTeX typing is found at \href{https://github.com/Johannesfjeldsaa/FYS3150_Latex}{github.com/Johannesfjeldsaa/FYS3150\_Latex}.

\end{document}