\documentclass[../main_proj3.tex]{subfiles}

\graphicspath{{\subfix{Figures/}}}

\begin{document}
\section{Introduction}

The purpose of this report template is two-fold: to communicate how to write a scientific report on a numerical topic, while simultaneously provide a concrete ``minimal working example'' of such a report. As our example topic we will look at an implementation of something rather dull and simple, namely the \textit{midpoint rule for integration}.

Throughout this example we will provide (hopefully) pedagogical commentary on report writing and structure, while at the same time present the midpoint rule for integration in a proper manner. To avoid confusion, we will from now on put pedagogical commentary in \textit{italics}. The next paragraph could have been part of the report, but we will leave it as a pedagogical comment: \textit{Writing reports, or papers, is a fundamental part of the scientific enterprise. It is vital to properly communicate precisely what has been done, what the results are and their implications. The motivation for this is to make the work understood and reproducible, so that others can both check and build on your work.}

\textit{The main purpose of the introduction section is to provide context and motivation for the work, like we have done above. It is also common --- and quite useful --- to use the last paragraph of the introduction to outline the rest of the report, to tell the reader what they should expect in the different sections. We will do this next.}

In Section \ref{sec:methods} we describe the mathematical background and formulate a concrete algorithm which can be implemented in any programming language. A selection of results from a validation test are presented in Section \ref{sec:results_and_discussion}. Here we also discuss our implementation of the algorithm in more detail. Finally, in Section \ref{sec:conclusion} we provide a short summary and outlook. 

All code developed for this report is available via a GitHub repository \href{github.com/Johannesfjeldsaa/FYS3150}{https://github.com/Johannesfjeldsaa/FYS3150}, and \LaTeX typing is found at \href{https://github.com/Johannesfjeldsaa/FYS3150_Latex}{github.com/Johannesfjeldsaa/FYS3150\_Latex}. 

\end{document}