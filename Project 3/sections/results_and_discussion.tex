\documentclass[../main_proj3.tex]{subfiles}

\graphicspath{{\subfix{Figures/}}}

\begin{document}


\section{Results and discussion}\label{sec:results_and_discussion}

To test the midpoint rule algorithm, we perform the integration of $f(x)$ using different choices for the number of subintervals. The results are listed in Table~\ref{tab:midpointruletab}.\footnote{A general style recommendation is to avoid having vertical lines in tables. There are of course exceptions, but in most cases vertical lines will make a table less readable.} 
We note that our implementation reproduces the analytical results to four digits precision when the integration range is divided into $n = 10^4$ subintervals. This indicates that that our implementation of the algorithm is correct.
%
\begin{table}[h!]
    \centering
    \caption{Approximate values for the integral of $f(x) = x^3$ on the interval $[0,1]$, as obtained with the midpoint rule with different numbers of integration subintervals.}
    \begin{tabular}{c@{\hspace{1cm}} c}
        \hline
        Number of subintervals & Integral value \\
        \hline
        $10^1$  &  0.3086 \\
        $10^2$  &  0.2550 \\
        $10^3$  &  0.2505 \\
        $10^4$  &  0.2500 \\
        % $10$  &  0.3086 \\
        % $100$  &  0.2550 \\
        % $1 000$  &  0.2505 \\
        % $10 000$  &  0.2500 \\
        \hline
    \end{tabular}\label{tab:midpointruletab}
\end{table}

In Figure~\ref{fig:rel_err} we show the relative error as a function of the number of subintervals $n$. We see that $\log_{10}(\epsilon)$ decreases linearly with $\log_{2}(n)$. From this, it should be possible to extract the convergence rate of our implementation of the midpoint rule. From a theoretical point of view we know that the midpoint rule should have a convergence rate of $\mathcal{O}(h^2)$. To properly verify our implementation, we should have estimated the convergence rate from our results and compared it to this theoretical rate. Without doing so, we cannot know that the our implementation of the algorithm is correct, even though we have seen that the numerical approximation converges to the correct answer in Table~\ref{tab:midpointruletab}.


\textit{Note especially how we reference both the table and the figure with a short explanation of their content. Always do this! In the figure/table captions we can also add additional information, such as information about how the figure/table was produced. You can also do this in the main text if you like. When writing the figure/table captions, keep in mind the general rule of thumb that an expert on the topic should be able to understand the gist of your report simply by reading the abstract and look at the figures/tables and read their corresponding captions.}

\textit{Although this is a somewhat silly example, please note the following: We are to-the-point in our discussion of the results, and we only make strong claims about what we are actually certain about. In the discussion it is important to try to be as concise as possible --- long paragraphs that only make very general points are typically of limited interest. Note that we also highlight aspects of our analysis that could have been improved and that might form a topic for future work.}

\end{document}