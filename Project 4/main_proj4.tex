\documentclass[12pt,a4paper,twocolumn]{article}
\usepackage[english]{babel}
\usepackage[utf8]{inputenc}
\usepackage[T1]{fontenc}

% Dokumentformatering:
\setlength\parindent{0pt} %ingen innrykk
\usepackage[parfill]{parskip} % Mellomrom mellom avsnitt
% Endre str
\usepackage[a4paper,top=3cm,bottom=2cm,left=1.5cm,right=1.5cm,marginparwidth=1.75cm]{geometry} % utnytter større del av arket.
\usepackage{appendix}
\usepackage{multicol}
\usepackage{titlesec}
\titleformat{\chapter}[display]
  {\normalfont\huge\bfseries}{\chaptertitlename\ \thechapter}{20pt}{\Huge}
\titlespacing*{\chapter}{0pt}{-70pt}{20pt} % reduce headspace of the chapter title
\setcounter{secnumdepth}{3} % Set the depth of section numbering to subsubsections



% Matte og symboler
\usepackage{lmodern}
\usepackage{textcomp, gensymb}
\usepackage{amssymb}
\usepackage{amsmath}
\usepackage{mathtools}
\usepackage{physics}
\usepackage{cancel} % Gjennomstreking i likninger etc
\usepackage{amsthm}
\usepackage{caption}
\usepackage{dirtytalk}
\usepackage{cancel}

% Pseudocode
\usepackage{algorithm}
\usepackage{algpseudocode}

% Håndterer grafikk:
\usepackage{graphicx}
\graphicspath{{Figurer/}} % Henter grafikk fra mappen "Figurer"
\usepackage{wrapfig}
%\setlength{\intextsep}{2pt} % adjust vertical space above and below the wrapfigure
\usepackage{subcaption} % eller subfigure
\usepackage[font=small,labelfont=bf]{caption} % Mindre figurtekst, bold på figur numerering
\usepackage{transparent} % Kan gjøre tekst mere gjenomsiktig
\usepackage{listings}
\usepackage{paralist}
\usepackage{tikz}
\usetikzlibrary{matrix}
\usetikzlibrary{fit}
\usetikzlibrary{intersections}
\usepackage{multirow}
\usepackage{pdfpages} % sett in pdf i dokumentet
\usepackage{subcaption} 
\usepackage{tabularx}
\usepackage{tabularray} % beste tabellpakke
\usepackage{pgfplots}
\usepgfplotslibrary{fillbetween}
\pgfplotsset{width=10cm,compat=1.9}
\newcommand{\tabitem}{~~\llap{\textbullet}~~} % for kulepunkter i tabeller uten vspace over dem


% Håndterer fine tabeller:
\usepackage{booktabs}
\usepackage{multirow}

% For å hente inn pdfsider. 
\usepackage{pdfpages}

% Håntere kilder med hyperlinker
\usepackage{csquotes}
\usepackage{hyperref}
\hypersetup{
    colorlinks=true,        % Colored links instead of frames
    linkcolor=blue,         % Color for internal links
    citecolor=red,          % Color for citations
    urlcolor=red            % Color for external links
}
\usepackage[style=numeric, sorting=none]{biblatex} 
\addbibresource{sources.bib}
\usepackage{lastpage} 
\usepackage[printonlyused,withpage]{acronym}
\usepackage{footmisc}




% Egne funksjoner
\newcommand{\frontmatter}{\cleardoublepage \pagenumbering{roman}}
\newcommand{\mainmatter}{\cleardoublepage \pagenumbering{arabic}}

\usepackage{caption}
\captionsetup{%
    ,format=hang
    ,justification=raggedright
    ,singlelinecheck=false
    ,figureposition=top
    }


\makeatletter
\AtBeginDocument{%
  \renewcommand*{\AC@hyperlink}[2]{%
    \begingroup
      \hypersetup{hidelinks}%
      \hyperlink{#1}{#2}%
    \endgroup
  }%
}
\makeatother





% Håndterer multi-fil oppsett:
\usepackage{xr}
\usepackage{subfiles}
\externaldocument{\subfix{main}}



\begin{document}

\twocolumn[
\begin{@twocolumnfalse}
\title{
The two-dimensional ising model:  \\
\huge Steady state solutions of paramagnets using markov-chain monte-carlo sampling 
}      % self-explanatory
\author{by: Johannes Fjeldså}          % self-explanatory
\date{\today}                             % self-explanatory

\maketitle 

%This is how we create an abstract section.
\begin{abstract}
    \noindent    
\end{abstract}

\end{@twocolumnfalse}
]


% ===============
% Introduction
% ===============

\subfile{sections/introduction}


% =====================
% Methods and theory
% =====================
\section{Methods and theory \protect\footnote{
Theory based on the course compendium and project description \cite{prosjekttbeskrivelse3}.
}}\label{sec:methods_and_theory}
\subfile{sections/methods_and_theory}



% ==========
% Results
% ==========
\section{Results and discussion}\label{sec:results_and_discussion}

\subfile{sections/results_and_discussion}


% =============
% Conclusion 
% =============

\subfile{sections/conclusion}

% ======
% Bib
% ======

\newpage 
\twocolumn[
\printbibliography
]
\newpage
\onecolumn

\appendix


\section{Figures of interacting particles}

Figure \ref{fig:phase-space} and \ref{fig:two_particles_3d} are additional results used to visualize the verification of code functionality. 

\begin{figure}[h!]
    \centering
    \includegraphics[width=.9\linewidth]{Project 3/figures/two_particles_v-vs-r.png}
    \caption{The phase-space plots for two particles. Upper row shows the velocity in the x plane as a function of x position and lower shows corresponding for y.}
    \label{fig:phase-space}
\end{figure}

\begin{figure}[h!]
    \centering
    \includegraphics[width=.9\linewidth]{Project 3/figures/two_particles_3d.png}
    \caption{The particle position in 3d. The particles are displayed in separate plots for illustration proposes but are simulated in the same Penning trap simultaneously. }
    \label{fig:two_particles_3d}
\end{figure}

\end{document}
