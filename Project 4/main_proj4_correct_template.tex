% USEFUL LINKS:
% -------------
%
% - UiO LaTeX guides:          https://www.mn.uio.no/ifi/tjenester/it/hjelp/latex/
% - Mathematics:               https://en.wikibooks.org/wiki/LaTeX/Mathematics
% - Physics:                   https://ctan.uib.no/macros/latex/contrib/physics/physics.pdf
% - Basics of Tikz:            https://en.wikibooks.org/wiki/LaTeX/PGF/Tikz
% - All the colors!            https://en.wikibooks.org/wiki/LaTeX/Colors
% - How to make tables:        https://en.wikibooks.org/wiki/LaTeX/Tables
% - Code listing styles:       https://en.wikibooks.org/wiki/LaTeX/Source_Code_Listings
% - \includegraphics           https://en.wikibooks.org/wiki/LaTeX/Importing_Graphics
% - Learn more about figures:  https://en.wikibooks.org/wiki/LaTeX/Floats,_Figures_and_Captions
% - Automagic bibliography:    https://en.wikibooks.org/wiki/LaTeX/Bibliography_Management  (this one is kinda difficult the first time)
%
%                              (This document is of class "revtex4-1", the REVTeX Guide explains how the class works)
%   REVTeX Guide:              http://www.physics.csbsju.edu/370/papers/Journal_Style_Manuals/auguide4-1.pdf
%
%
% COMPILING THE .pdf FILE IN THE LINUX TERMINAL
% ---------------------------------------------
%
% [terminal]$ pdflatex report_example.tex
%
% Run the command twice, always.
%
% When using references, footnotes, etc. you should run the following chain of commands:
%
% [terminal]$ pdflatex report_example.tex
% [terminal]$ bibtex report_example
% [terminal]$ pdflatex report_example.tex
% [terminal]$ pdflatex report_example.tex
%
% This series of commands can of course be gathered into a single-line command:
% [terminal]$ pdflatex report_example.tex && bibtex report_example.aux && pdflatex report_example.tex && pdflatex report_example.tex
%
% ----------------------------------------------------



% \documentclass[english,notitlepage,reprint,nofootinbib]{revtex4-2}  % defines the basic parameters of the document
\documentclass[english,notitlepage,reprint,nofootinbib]{revtex4-2}  % defines the basic parameters of the document
% If you want a single-column, remove "reprint"

% Allows special characters (including æøå)
\usepackage[utf8]{inputenc}
\usepackage[english]{babel}

% Note that you may need to download some of these packages manually, it depends on your setup.
% It may be usefult to download TeXMaker, because it includes a large library of the most common packages.

\usepackage{physics,amssymb}  % mathematical symbols (physics imports amsmath)
\usepackage{amsmath}
\usepackage{graphicx}         % include graphics such as plots
\usepackage{xcolor}           % set colors
\usepackage{hyperref}         % automagic cross-referencing
\usepackage{listings}         % display code
\usepackage{subfigure}        % imports a lot of cool and useful figure commands
% \usepackage{float}
%\usepackage[section]{placeins}
\usepackage{algorithm}
\usepackage[noend]{algpseudocode}
\usepackage{subfigure}
\usepackage{tikz}
\usepackage{booktabs}
\usepackage{varwidth}
% defines the color of hyperref objects
% Blending two colors:  blue!80!black  =  80% blue and 20% black
\hypersetup{ % this is just my personal choice, feel free to change things
    colorlinks,
    linkcolor={red!50!black},
    citecolor={blue!50!black},
    urlcolor={blue!80!black}
    }


% Håndterer multi-fil oppsett:
\usepackage{xr}
\usepackage{subfiles}
\externaldocument{\subfix{main}}

% ===========================================


\begin{document}

\title{
The two-dimensional Ising model:\\
An numerical investigation of distribution functions and phase transitions
}      % self-explanatory
\author{by: Johannes Fjeldså}          % self-explanatory
\date{\today}                             % self-explanatory
\noaffiliation                            % ignore this, but keep it.

%This is how we create an abstract section.
\begin{abstract}
    In this study, the Metropolis algorithm is used to perform numerical investigations of distribution functions and phase transitions for the two-dimensional Ising model. The method shows promising results when benchmarking against analytical results for the $2\times 2$ model, indicating that it is a valid approach. Based on convergence of expectation values, we find an equilibration time of $10^{4}-10^{5}$ Monte Carlo cycles. This burn-in period is therefore used for further investigations. Further, we use MCMC sampling to estimate the Boltzmann distributions of temperatures above and below the critical Curie temperature ($T_c$), revealing a less stable state for higher temperatures. The magnetic phase transitions occurring at $T_c$ are further investigated, and we estimate a $T_c$ for an infinite model to be $\hat{T}_c=2.251~J/k_B$ fairly close to the analytical value of $T_c=2.269~J/k_B$.
\end{abstract}
\maketitle


% ===========================================
% Introduction
% ===========================================

\subfile{sections/introduction}


% ===========================================
% Methods
% ===========================================

\subfile{sections/methods}


% ===========================================
% Results and discussion
% ===========================================

\subfile{sections/results_and_discussion}


% ===========================================
% Conclusion
% ===========================================

\subfile{sections/conclusion}







% ===========================================
\newpage
\onecolumngrid
% \bibliographystyle{apalike}
\bibliographystyle{unsrt}
\bibliography{sources}

% ===========================================
% Appendix
% ===========================================
\newpage
\subfile{sections/appendix}
\end{document}