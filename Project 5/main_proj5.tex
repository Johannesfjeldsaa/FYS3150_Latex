% USEFUL LINKS:
% -------------
%
% - UiO LaTeX guides:          https://www.mn.uio.no/ifi/tjenester/it/hjelp/latex/
% - Mathematics:               https://en.wikibooks.org/wiki/LaTeX/Mathematics
% - Physics:                   https://ctan.uib.no/macros/latex/contrib/physics/physics.pdf
% - Basics of Tikz:            https://en.wikibooks.org/wiki/LaTeX/PGF/Tikz
% - All the colors!            https://en.wikibooks.org/wiki/LaTeX/Colors
% - How to make tables:        https://en.wikibooks.org/wiki/LaTeX/Tables
% - Code listing styles:       https://en.wikibooks.org/wiki/LaTeX/Source_Code_Listings
% - \includegraphics           https://en.wikibooks.org/wiki/LaTeX/Importing_Graphics
% - Learn more about figures:  https://en.wikibooks.org/wiki/LaTeX/Floats,_Figures_and_Captions
% - Automagic bibliography:    https://en.wikibooks.org/wiki/LaTeX/Bibliography_Management  (this one is kinda difficult the first time)
%
%                              (This document is of class "revtex4-1", the REVTeX Guide explains how the class works)
%   REVTeX Guide:              http://www.physics.csbsju.edu/370/papers/Journal_Style_Manuals/auguide4-1.pdf
%
%
% COMPILING THE .pdf FILE IN THE LINUX TERMINAL
% ---------------------------------------------
%
% [terminal]$ pdflatex report_example.tex
%
% Run the command twice, always.
%
% When using references, footnotes, etc. you should run the following chain of commands:
%
% [terminal]$ pdflatex report_example.tex
% [terminal]$ bibtex report_example
% [terminal]$ pdflatex report_example.tex
% [terminal]$ pdflatex report_example.tex
%
% This series of commands can of course be gathered into a single-line command:
% [terminal]$ pdflatex report_example.tex && bibtex report_example.aux && pdflatex report_example.tex && pdflatex report_example.tex
%
% ----------------------------------------------------



% \documentclass[english,notitlepage,reprint,nofootinbib]{revtex4-2}  % defines the basic parameters of the document
\documentclass[english,notitlepage,reprint,nofootinbib]{revtex4-2}  % defines the basic parameters of the document
% If you want a single-column, remove "reprint"

% Allows special characters (including æøå)
\usepackage[utf8]{inputenc}
\usepackage[english]{babel}

% Note that you may need to download some of these packages manually, it depends on your setup.
% It may be usefult to download TeXMaker, because it includes a large library of the most common packages.

\usepackage{physics,amssymb}  % mathematical symbols (physics imports amsmath)
\usepackage{amsmath}
\usepackage{graphicx}         % include graphics such as plots
\usepackage{xcolor}           % set colors
\usepackage{hyperref}         % automagic cross-referencing
\usepackage{listings}         % display code
\usepackage{subfigure}        % imports a lot of cool and useful figure commands
% \usepackage{float}
%\usepackage[section]{placeins}
\usepackage{algorithm}
\usepackage[noend]{algpseudocode}
\usepackage{subfigure}
\usepackage{tikz}
\usepackage{booktabs}
\usepackage{varwidth}
% defines the color of hyperref objects
% Blending two colors:  blue!80!black  =  80% blue and 20% black
\hypersetup{ % this is just my personal choice, feel free to change things
    colorlinks,
    linkcolor={red!50!black},
    citecolor={blue!50!black},
    urlcolor={blue!80!black}
    }


% Håndterer multi-fil oppsett:
\usepackage{xr}
\usepackage{subfiles}
\externaldocument{\subfix{main}}

% ===========================================


\begin{document}

\title{
The double slit experiment:\\
An numerical investigation of the time-dependent Schr\"odinger equation
}      % self-explanatory
\author{by: Johannes Fjeldså}          % self-explanatory
\date{\today}                             % self-explanatory
\noaffiliation                            % ignore this, but keep it.

%This is how we create an abstract section.
\begin{abstract}
\noindent In this study, the Crank-Nicolson discretization is used to create a simulator of a particle in a double-slit experiment setting. The method shows a high degree of computational stability, but longer simulations should be conducted to conclude on whether or not errors accumulate or are recovered. Building on this, we use the same simulator but with shorter simulation times to investigate the behavior of a particle modeled with the time-dependent Schr\"odinger equation. The simulator shows the ability to replicate the expected wave nature, such as interference patterns from a propagation to a slit wall.
    \end{abstract}
\maketitle


% ===========================================
% Introduction
% ===========================================

\subfile{sections/introduction}


% ===========================================
% Methods
% ===========================================

\subfile{sections/methods}


% ===========================================
% Results and discussion
% ===========================================

\subfile{sections/results_and_discussion}


% ===========================================
% Conclusion
% ===========================================

\subfile{sections/conclusion}







% ===========================================
\newpage
\onecolumngrid
% \bibliographystyle{apalike}
\bibliographystyle{unsrt}
\bibliography{sources}

% ===========================================
% Appendix
% ===========================================
\newpage
\subfile{sections/appendix}
\end{document}