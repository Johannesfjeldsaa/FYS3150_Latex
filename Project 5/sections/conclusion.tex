\documentclass[../main_proj5e.tex]{subfiles}

\graphicspath{{\subfix{Figures/}}}

\begin{document}


\section{Conclusion}\label{sec:p5_conclusion}

In this study, we have investigated the Crank-Nicolson discretization of a scaled version of the time-dependent Schr\"odinger equation in 2 spatial dimensions. Firstly, we show how to discretize the equation before implementing a code that is able to simulate a propagating particle. The first simulations were made to ensure that the simulation conserves the total probability of the system. Results showed that the solving algorithm SuperLU had numerical convergence with deviations on the order of $10^{-14}$ only one order of magnitude larger than that of the c++ double used. Further, we saw that numerical instabilities had a larger prevalence during changes in the distribution resulting from incidents with the slit and walls. To evaluate the impact of these leaks, a longer simulation should be conducted. Further, we used three different slit wall setups to investigate the wave nature of the particle. By estimating the marginal distribution at $x=0.8$, the single-slit experiment shows that the distribution remains unimodal after slit impact. For double and triple slits we see the expected interference pattern, which has been seen in many experimental settings. These results show that the developed simulator is able to conserve the total probability in the system and is able to simulate the expected physics. For future work, one key investigation would be to look into more optimized computational algorithms, both in terms of CPU cost and memory usage.

\end{document}
