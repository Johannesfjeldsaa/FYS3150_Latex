\documentclass[../main_proj5.tex]{subfiles}

\graphicspath{{\subfix{Figures/}}}

\begin{document}
\section{Introduction}\label{sec:p5_Introduction}

Today, the wave-particle duality is well known. This is, however, a discovery that was first formally suggested 100 years ago by Louis de Broglie in his 1924 PhD thesis \cite{nolte100YearsQuantum2024}. Quickly following this was confirmed by the Davisson-Germer experiment in 1927, and the equation of motion for these particles, the Schr\"odinger equation, was suggested in 1926 by Erwin Schr\"odinger \cite{townsend2010quantum}. The Sch\"odinger equation yielded a computational model for representing the time-evolving quantum state of a physical system of non-relativistic particles. As opposed to particle equations of classical mechanics, the Schr\"odinger equation represents the quantum state as a wave function that is related to probability through Born's rule (postulated by Max Born in 1926 )\cite{prosjekttbeskrivelse5}. In this study, we explore the implications of the Schr\"odinger equation for the double-slit experiment by numerically propagating the equation through time in the radial plane. As a sanity check, we also investigate the numerical stability of the simulation by checking for probability conservation.

The study is structured as follows: In Section \ref{sec:p5_theory} we give an introduction to the Schr\"odinger equation and Born's Rule. Further, we show how a Crank-Nicolson discretization allows us to rewrite a scaled version of the equation as a matrix-vector equation. Lastly, the theory presents the initialization of the numerical solver and the experimental design for simulations. In section \ref{sec:p5_results_and_discussion}, the results from model simulations are shown and discussed. Lastly, we conclude on the proficiency of the Crank-Nicolson discretization for propagating partial differential equations, such as the Sch\"odinger equation, forward in time in Section \ref{sec:p5_conclusion}. Appendix \ref{app:p5_AppendixA} provides further details on derivations of the discretized equations. All code developed for this report is available via the UiO GitHub repository \href{https://github.uio.no/johannlf/FYS3150}{https://github.uio.no/johannlf/FYS3150}, and \LaTeX typesetting is found at \href{https://github.com/Johannesfjeldsaa/FYS3150_Latex}{https://github.com/Johannesfjeldsaa/FYS3150\_Latex}.

\end{document}
