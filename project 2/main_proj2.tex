\documentclass[english,notitlepage]{article}  % defines the basic parameters of the document
%For preview: skriv i terminal: latexmk -pdf -pvc filnavn


\usepackage[english]{babel}
\usepackage[utf8]{inputenc}
\usepackage[T1]{fontenc}

% Dokumentformatering:
\setlength\parindent{0pt} %ingen innrykk
\usepackage[parfill]{parskip} % Mellomrom mellom avsnitt
% Endre str
\usepackage[a4paper,top=3cm,bottom=2cm,left=1.5cm,right=1.5cm,marginparwidth=1.75cm]{geometry} % utnytter større del av arket.
\usepackage{appendix}
\usepackage{multicol}
\usepackage{titlesec}
\titleformat{\chapter}[display]
  {\normalfont\huge\bfseries}{\chaptertitlename\ \thechapter}{20pt}{\Huge}
\titlespacing*{\chapter}{0pt}{-70pt}{20pt} % reduce headspace of the chapter title
\setcounter{secnumdepth}{3} % Set the depth of section numbering to subsubsections



% Matte og symboler
\usepackage{lmodern}
\usepackage{textcomp, gensymb}
\usepackage{amssymb}
\usepackage{amsmath}
\usepackage{mathtools}
\usepackage{physics}
\usepackage{cancel} % Gjennomstreking i likninger etc
\usepackage{amsthm}
\usepackage{caption}
\usepackage{dirtytalk}
\usepackage{cancel}

% Pseudocode
\usepackage{algorithm}
\usepackage{algpseudocode}

% Håndterer grafikk:
\usepackage{graphicx}
\graphicspath{{Figurer/}} % Henter grafikk fra mappen "Figurer"
\usepackage{wrapfig}
%\setlength{\intextsep}{2pt} % adjust vertical space above and below the wrapfigure
\usepackage{subcaption} % eller subfigure
\usepackage[font=small,labelfont=bf]{caption} % Mindre figurtekst, bold på figur numerering
\usepackage{transparent} % Kan gjøre tekst mere gjenomsiktig
\usepackage{listings}
\usepackage{paralist}
\usepackage{tikz}
\usetikzlibrary{matrix}
\usetikzlibrary{fit}
\usetikzlibrary{intersections}
\usepackage{multirow}
\usepackage{pdfpages} % sett in pdf i dokumentet
\usepackage{subcaption} 
\usepackage{tabularx}
\usepackage{tabularray} % beste tabellpakke
\usepackage{pgfplots}
\usepgfplotslibrary{fillbetween}
\pgfplotsset{width=10cm,compat=1.9}
\newcommand{\tabitem}{~~\llap{\textbullet}~~} % for kulepunkter i tabeller uten vspace over dem


% Håndterer fine tabeller:
\usepackage{booktabs}
\usepackage{multirow}

% For å hente inn pdfsider. 
\usepackage{pdfpages}

% Håntere kilder med hyperlinker
\usepackage{csquotes}
\usepackage{hyperref}
\hypersetup{
    colorlinks=true,        % Colored links instead of frames
    linkcolor=blue,         % Color for internal links
    citecolor=red,          % Color for citations
    urlcolor=red            % Color for external links
}
\usepackage[style=numeric, sorting=none]{biblatex} 
\addbibresource{sources.bib}
\usepackage{lastpage} 
\usepackage[printonlyused,withpage]{acronym}
\usepackage{footmisc}




% Egne funksjoner
\newcommand{\frontmatter}{\cleardoublepage \pagenumbering{roman}}
\newcommand{\mainmatter}{\cleardoublepage \pagenumbering{arabic}}

\usepackage{caption}
\captionsetup{%
    ,format=hang
    ,justification=raggedright
    ,singlelinecheck=false
    ,figureposition=top
    }


\makeatletter
\AtBeginDocument{%
  \renewcommand*{\AC@hyperlink}[2]{%
    \begingroup
      \hypersetup{hidelinks}%
      \hyperlink{#1}{#2}%
    \endgroup
  }%
}
\makeatother





% Håndterer multi-fil oppsett:
\usepackage{xr}
\usepackage{subfiles}
\externaldocument{\subfix{main}}



\begin{document}

\title{
Solution to Project 1 in \\
\huge FYS3150 Computation Physics 
}      % self-explanatory
\author{by: Johannes Fjeldså}          % self-explanatory
\date{\today}                             % self-explanatory


\maketitle 
    
\textit{My github repo is found at: \href{https://github.com/Johannesfjeldsaa/FYS3150}{https://github.com/Johannesfjeldsaa/FYS3150}, \\
and \LaTeX typing is found at \href{https://github.com/Johannesfjeldsaa/FYS3150_Latex}{https://github.com/Johannesfjeldsaa/FYS3150\_Latex}}    

\tableofcontents


\newpage
\subfile{Problems/problem1}

\newpage
\subfile{Problems/problem2}

\newpage
\subfile{Problems/problem3}

\newpage
\subfile{Problems/problem4}

\newpage
\subfile{Problems/problem5}

\newpage
\subfile{Problems/problem6}



\section*{Problem 2}
We write equations using the LaTeX \texttt{equation} (or \texttt{align}) environments. Here is an equation with numbering
\begin{equation}\label{eq:newton}
    \vb{F} = \dv{\vb{p}}{t},
\end{equation}
and here is one without numbering:
\begin{equation*}
\oint_C \vb{F}\cdot \dd \vb{r} = 0.
\end{equation*}
Sometimes it is useful to refer back to a previous equation, like we're demonstrating here for equation \ref{eq:newton}.

We can include figures using the \texttt{figure} environment. Whenever we include a figure or table, we \textit{must} make sure to actually refer to it in the main text, e.g.\ something like this: ``In figure \ref{fig:rel_err} we show \ldots''. 
\begin{figure}%[h!]
    \centering %Centers the figure
    \includegraphics[scale=0.55]{imgs/rel_err.pdf} %Imports the figure.
    \caption{Write a descriptive caption here that explains the content of the figure. Note the font size for the axis labels and ticks --- the size should approximately match the document font size.}
    \label{fig:rel_err}
\end{figure}
Also, note the LaTeX code we used to get correct quotation marks in the previous sentence. (Simply using the \texttt{"} key on your keyboard will give the wrong result.) Figures should preferably be vector graphics (e.g.\ a \texttt{.pdf} file) rather than raster graphics (e.g.\ a \texttt{.png} file).

By the way, don't worry too much about where LaTeX decides to place your figures and tables --- LaTeX knows more than we do about proper document layout. As long as you label all your figures and tables and refer to them in the text, it's all good. Of course, in some cases it can be worth trying to force a specific placement, to avoid the figure/table appearing many pages away from the main text discussing it, but this isn't something you should spend time on until the very end of the writing process.


Next up is a table, created using the \texttt{table} and \texttt{tabular} environments. We refer to it by table \ref{tab:output_table}.
\begin{table}%[h!]
    \centering
    \caption{Write a descriptive caption here, explaining the content of your table.}
    \begin{tabular}{c@{\hspace{1cm}} c}
        \hline
        Number of points & Output \\
        \hline
        10 &  0.3086\\
        100 &  0.2550\\
        \hline
    \end{tabular}\label{tab:output_table}
\end{table}

Finally, we can list algorithms by using the \texttt{algorithm} environment, as demonstrated here for algorithm \ref{algo:midpoint_rule}.
\begin{algorithm}[H]
    \caption{Some algorithm}\label{algo:midpoint_rule}
    \begin{algorithmic}
        \State Some maths, e.g $f(x) = x^2$.  \Comment{Here's a comment}
        \For{$i = 0, 1, ..., n-1$}
        \State Do something here 
        \EndFor
        \While{Some condition}
        \State Do something more here 
        \EndWhile
        \State Maybe even some more math here, e.g $\int_0^1 f(x) \dd x$
    \end{algorithmic}
\end{algorithm}
   
\end{document}